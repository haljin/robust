\chapter{Robustness and testing}
We have discussed how the code is implemented and what mechanisms were used to ensure it works correctly. It is now time to evaluate and analyse whether the goal was achieved and the system is truly robust. To do that we used a variety of tools, different for both languages, as both are distinct and have their own qualities, requiring a slightly different approach.
\section{C\#, Pex and Unit Testing}
When developing the C\# version we have been using Code Contracts, as described in the previous section, in order to ensure correct behaviour of our code. However the contracts themselves are not a tool that removes errors --- it only helps to detect them, when they occur. That is we will get a \emph{ContractException} or an \emph{ArgumentException} in case, respectively, the post and pre conditions are not satisfied. This on its own does not stop the reasons for these conditions not being met. Thus, we need another tool that will help us identify, which situations read to erroneous inputs/outputs that violate the contracts.

Therefore, we need certain testing methods, in order to be able to locate the bugs. The most popular way of testing pieces of code is unit testing. Unit testing is a practice of testing the individual pieces of code (units) such as single methods or small sequences of methods.~\cite{MSDN} By comparing the received output with expected, with given input, we are able to determine whether the code works correctly. This works twice as well, when used in conjunction with Code Contracts, as any incorrect behaviour within the function will be caught by the contracts and an appropriate exception will be thrown.

Unit testing however is not a simple ordeal. They only test, what the developer has created them to test --- in other words, they will not test the code under circumstances which the developer did not anticipate. That is a common problem as usually the same developer that wrote the tested code is writing the tests themselves, therefore if he did not anticipate an erroneous situation in the code, he will not test against it. As such, if the situation actually occurs in production, it will be much too late.

There are certain tools that are meant to assist with that predicament and assure that the code is tested against all possible conditions. One of such tools is Pex, which is an on-going research project at Microsoft that aims to create a comprehensible white-box testing tool that will automatically generate all the unit tests necessary to fully test the code against all possible situations.~\cite{MSDN} We have decided to employ Pex to our code in order to test it thoroughly.

Pex exploration has been run against all methods within the code. Unfortunately, the results proved to be rather underwhelming. Due to the nature of our code, Pex was unable to generate any meaningful tests for the majority of the methods – and it also struggled with those it managed to explore. Only \emph{AddCoin} and \emph{EjectCoin} methods of the \emph{CoinManager} could have been tested properly, and the achieved code coverage results were only $46.90\%$ coverage for the \emph{CoinManager} class alone.

The results were poor, mostly due to the fact, that much of our methods and classes’ behaviour depends on the state of the database --- for instance, \emph{EjectProduct} will behave differently if the product is, or is not in stock. Therefore, Pex is unable to generate meaningful XML files, without a lot of intervention from our side. Pex also had problems with creating objects of the Product class, to do anything meaningful with the \emph{StockManager} class’ methods. On the other hand, for methods in which the state of the XML is either irrelevant (such as \emph{AddCoin}) or other auxiliary methods (such as override of \emph{Product’s} Equals()), Pex did a good job of mapping the code and generating appropriate unit tests. Figure~\ref{fig:pex} shows the examples of successful exploration.

\begin{figure}[!ht]
\centering
\begin{minipage}{\linewidth}
\centering\large \includegraphics[width=\textwidth]{pictures/pex1.png}
\subcaption{Coin.ToValue}\label{fig:coinToValue}
\end{minipage} \\
\begin{minipage}{\linewidth}
\centering\large \includegraphics[width=\textwidth]{pictures/pex2.png}
\subcaption{CoinManager.AddCoin}\label{fig:addCoin}
\end{minipage}
\caption{Pex exploration results}\label{fig:pex}
\end{figure}

Additionally, our code contains heavy usage of LINQ expressions that are used to query the database XML file, via XML to LINQ. Pex seems to struggle with LINQ expressions used within the code and cannot correctly map all the possibilities. Especially so, when LINQ is actually contained within the contracts.

What we can say in favour of Pex, is that it is compatible with Code Contracts, the contracts themselves actually ,,helping'' Pex with exploration and its exceptions are generally recognized by Pex when performing exploration.

Unfortunately, using Pex was not enough to prove our code is robust and reliable, therefore we had to use other means to ensure that this is indeed the case. To supplement the unit tests generated by Pex, we have created a number of custom unit tests to cover the other methods that Pex was unable to map.

We have created 36 additional unit tests that were meant to cover all the missed out scenarios for all the other methods. We have tested all parts of our application, save for the GUI, as this is this is an additional component that is not crucial --- it is the main engine of the application that matters most. To evaluate these tests we have used the Code Coverage tool included in Visual Studio 2010, with which we were able to evaluate our tests just as well as the Pex generated ones. Figure~\ref{fig:cov} shows the coverage results. The relevant classes are highlighted with red boxes, as the other are helper classes, auto-generated by Visual Studio in regards to LINQ and contracts.

\begin{figure}[htbp]
\centering
\includegraphics[width = \textwidth]{pictures/codeCoverage.PNG}
\caption{Code coverage results}\label{fig:cov}
\end{figure}

As can be seen the coverage obtained in the relevant areas oscillates around $90\%$, with \emph{VendMachine} being lowest at $75.84\%$. Still, the tool allows us to evaluate at which portions of code were actually not covered. It turns out the non-covered parts are actually ,,partially covered'' and all of them are LINQ expressions within contracts. As such, they are incredibly hard to fully test in order to achieve maximum coverage. 

With these considerations we have decided that our program is indeed robust and should withstand even the most unusual situations --- and if not, the contract exceptions would help us identify and quickly correct the issue that occurred.

\section{Erlang and Quickcheck}
Testing Erlang applications is a bit different. While a framework for unit tests does indeed exist, there is a much better solution available that is used commercially throughout the industry, where Erlang is applied (for instance Motorola uses this method for verification of Erlang code). This solution is called Quickcheck.

Developed at first at University of Gothenburg, this solution is now provided commercially by a company called QuviQ.~\cite{QuviQ} It is a new and innovative approach to testing, where instead of testing specific scenarios created by unit tests, certain properties that must always hold within the system are tested instead. Quickcheck is available primarily for Erlang, but also for C. It comes in two versions --- Mini, which is free, and Full, which requires a purchased license.~\cite{QuviQ} As such, we will be only using the Mini version, but it is more than enough for our purposes.

The principles of Quickcheck are quite simple --- the libraries provided contain several function and macros for creating everything that is necessary --- although learning Quickcheck may be quite complicated at the beginning, almost as learning Erlang itself. What is needed to perform tests is two things, fundamental to property-based testing --- properties themselves and generators.

The generators are responsible for generating the input data. They are created using Quickcheck’s inbuilt simple type generators and are meant to be able to generate values from the full range of possible (correct --- as we do not program defensively in Erlang! Incorrect input will result in a crash that is expected and handled by supervisors. Correct data however, should not crash) input values.

Properties are conditions which always hold true within the system. For instance, after a correct purchase, the product the user bought must be located in the case or that on a cancelled transaction it should not. Quickcheck server will then use the generators to generate random inputs within the correct range. The result will be tested against the given property. Figure~\ref{fig:qcheck} illustrates the process.

\begin{figure}[htbp]
\centering
\includegraphics[width = 0.7\textwidth]{pictures/quickcheck.PNG}
\caption{Quickcheck paradigm}\label{fig:qcheck}
\end{figure}

It does not perform only one test however --- the default is 100 tests. As the test progress and pass, the generated data is increased in complexity. That is why defining a good generator is important, as Quickcheck must ,,know'' how to generate more complex data.

However the true strength of Quickcheck lies in the case when a test fails. When that happens a process called ,,shrinking'' occurs, where the generators are queried to simplify the data that they provided in the failing case. The server will attempt to find the least complex example that still causes the property to fail. This is illustrated by Figure~\ref{fig:shr}.

\begin{figure}[htbp]
\centering
\includegraphics[width = 0.7\textwidth]{pictures/shrinking.PNG}
\caption{Shrinking}\label{fig:shr}
\end{figure}

Thanks to this, we are able to retrieve the simplest test case, which makes debugging much easier, as reproducing very complicated data sets might prove to be time-consuming, while in the same complicated sets it is much more likely errors will occur.
\subsection{Generators}
For the purpose of testing our vending machine application we have created several generators. Listing them would be pointless as there is a large number of them, several of which are just auxiliary generators for other generators. We can however show a sample generator and explain how it works.
\begin{lstlisting}[style=myErlang,captionpos=b,caption={Stock generator}]
prod_generator() ->
    Products = [cola,chips,candy,faxe_kondi,liquorice,some_random_crap,jelly_beans],
    ?SUCHTHAT(List, 
	      non_empty(list(#product{name = oneof(Products), price = choose(2, 40), ammount = choose(1,10)})),
	      lists:all(fun(E) ->
				E=<1
			end, 
			lists:map(fun(El) ->
					  lists:foldl(fun(El2, Sum) ->
							      case El2#product.name of
								  El ->
								      Sum +1;
								  _ ->
								      Sum
							      end
						      end, 0, List)
				  end, Products))).
\end{lstlisting}
It is quite a complicated generator but we will attempt to clarify how it works. First of all, the generator is built out of two parts, within the \texttt{?\color{cyan}{SUCHTHAT}} macro. In principle, this macro ensures, that when a value is generated by the generator it satisfies a given condition, otherwise the value is discarded. In this example a value is generated by 
\begin{lstlisting}[style=myErlang,captionpos=b,caption={The generator itself}]
non_empty(list(#product{name = oneof(Products), price = choose(2, 40), ammount = choose(1,10)})) 
\end{lstlisting}
and then assigned to the variable \emph{List}, which is used to evaluate the long condition. The generator itself is quite simple. It generates a list (\texttt{\color{blue}{list}()}) of product records (as denoted by \texttt{\#\color{blue}{product{}}}), that is not empty (\texttt{\color{blue}{non\_empty}()}), so there is at least one product on the list. For each of those products, its name is chosen at random, from the list defined above the macro, its price is within the range of 2 and 40 (\texttt{\color{blue}{choose}()}) and the amount of this product is between 1 and 10. The functions mentioned in the brackets are in-built Quickcheck basic generators, which we use to build a more complex type. 

The long condition that follows in the \texttt{?\color{cyan}{SUCHTHAT}} macro simply ensures, that there are no two products with the same name in stock, as such case is impossible (\emph{dets} keys are unique and that is how the stock is stored).

\begin{lstlisting}[style=myErlang,captionpos=b,caption={Sample value created by generator}]
[{product,chips,12,1},
 {product,some_random_crap,10,5},
 {product,candy,9,2},
 {product,jelly_beans,4,1},
 {product,cola,9,1},
 {product,liquorice,39,9}]
\end{lstlisting}

This list was no doubt generated for large generation size, as it is quite long. In the end a generator must always return either a Quickcheck generator function (or a combination thereof) or a constant value, that will become a constant generator.

One last consideration with Quickcheck generators is function call generators. Consider the following generator:
\begin{lstlisting}[style=myErlang]
insertion_generator(Choices, Stock) ->
    {call, vm_control, choose_product, [ProdName]} = lists:last(Choices),
    Product = lists:keyfind(ProdName, #product.name, Stock), 
    vector(Product#product.price, {call, vm_control, insert_coin, [kr1]}).
\end{lstlisting}
It generates a list (\texttt{\color{blue}{vector}()} generates a list of a set size) of tuples of the form {\emph{call}, \emph{vm\_control}, \emph{insert\_coin}, [kr1]} that  would evaluate to calling the \emph{insert\_coin} function with a 1 kroner coin as an argument. This is highly useful, as in case of a failure, Quickcheck will print out the counter-example which failed. In this case, the evaluated value of the function would not provide us with any information, especially as it is a call towards a separate process. Getting a symbolic function call however, allows us to immediately pin-point what exactly happened in the system that caused its crash! 
\subsection{Properties}
We have created several properties in order to test the correctness of our system’s behavior. Developing of the properties should always be done on the system specification and we have followed this guideline. Before we elaborate on the properties that have been created, let us look closer on how does a Quickcheck property work, looking at an example.
\begin{lstlisting}[style=myErlang]
purchase_property()->
    ?FORALL({{Choices, Insertion, Stock}, Coins}, 
	    {transaction_generator(), coin_generator()},
	    begin
		vm_coin:generate_dets(Coins),
		vm_stock:generate_dets(Stock),
		timer:sleep(100),
		vm_sup:start(),
		{call, vm_control, choose_product, [BoughtItem]} = lists:last(Choices),
		lists:foreach(fun(El) -> 
				      eval(El),
				      timer:sleep(10) 
			      end, Choices ++ Insertion),
		timer:sleep(200),		
		Result = lists:keymember(BoughtItem, 1, vm_case:get_all()),
		vm_sup:stop(),
		timer:sleep(400),
		file:delete("stock.db"),
		file:delete("coin.db"),	
		Result
	    end).
\end{lstlisting}
Every Quickcheck property are based on the \texttt{?\color{cyan}{FORALL}} macro, meaning simply that for all values generated by given generators a specific property should hold. In this example we are testing the correctness of the property, that after a successful purchase transaction, the product is present in the case. We use two generators: \texttt{\color{blue}{transaction\_generator}()} and \texttt{\color{blue}{coin\_generator}()} which generate an appropriate sequence of actions together with the machines stock and coins contained in the machine, respectively. The values generated by these two generators are bound to the appropriate variables that are then used in the property.

The property is first executing a sequence of actions that is in order:
\begin{itemize}
\item generating a \emph{dets} file with the stock,
\item generating a \emph{dets} file with the coins,
\item activating the vending machine,
\item evaluating all symbolic function calls in order to execute all the generated actions,
\item clean-up --- deleting temporary \emph{dets} files and stopping the machine.
\end{itemize}

This property allows us to execute many tests of various complexity (meaning varying stock size, item prices and the amount of ,,indecision'' that is choose product actions performed before settling down on an item an inserting coins).

Other properties that we have developed are:
\begin{description}
\item[Cancel property] --- on cancellation the product should not be in the case and all money inserted has to be returned.
\item[Change property] --- upon inserting more money than the product is worth, correct change is returned, that is either the appropriate amount or as much as the machine can give.
\item[No stock property] --- the user may not proceed with purchase of a product that is out of stock.
\end{description}

Together these properties test against all the limitations and prerequisites described in the specification. Thanks to Quickcheck we were able to verify that our code is indeed correct and several test runs of default 100 tests have concluded successfully assuring us that for correct input the machine behaves appropriately.
\subsection{Other}
Thanks to Quickcheck we have successfully verified that under correct circumstances, the system we developed works correctly. What about incorrect circumstances however? This is where OTP’s robustness mechanisms come to play. Thanks to the supervision trees in case an incorrect input is entered the process that receives that input will simply crash, as we do not program defensively. The crash however is not fatal as the supervisors will promptly restart the crashing process and the system goes back to normal. The chances of the same incorrect input appearing again are in fact very low (as they could, for instance, result from interference on communication between various actual hardware components) and as such this is a valid approach, that is widely used in Erlang solutions.~\cite{Armstrong2003}

With help from these powerful mechanisms we have a really robust and powerful system that can withstand many problems while providing maximum service uptime. Of course that does not mean that it is perfect and flawless --- but it does mean that it is highly reliable and any inconveniences are likely to be very temporary.

