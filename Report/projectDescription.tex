\chapter{Project description}
To demonstrate the main differences in the two aforementioned approaches we decided to re-create a simple system in both of these technologies. The system we have chosen was meant to be simple as to focus mainly on the implementation  and demonstration of the techniques, while not bogging us down in irrelevant technicalities.  

We chose to design a simple vending machine software. A vending machine has a certain stock of items and a ,,wallet'' containing coins for change. A user can simply approach the machine and select the desired product by pressing an appropriate button. The machine responds with informing the user of the cost of the item. He is also warned in case of the given product being out of stock. Afterwards the user can start inserting the money into the machine and once he has put enough, the product is ejected into a case on the bottom of the machine along with change, if necessary. The machine will give the highest possible nominator as change. It also accepts and returns only Danish coins. In a scenario where the machine does not contain enough coins to give change to the user, it will attempt to return as much money as possible, however not exceeding the total value to be returned --- in other words, if the machine contains only 20 Kr coins and has to return 15 Kr, it will return no change at all. A warning must be displayed to the user, so he does not presume the machine is broken, merely out of change. The returned coins are dropped into a coin case also at the bottom of the machine.

Any money inserted into the machine is immediately added into the money pool of the machine. 

The machine also comes with a ,,cancel'' button that the user can press at any moment before the project is ejected (that is, before inserting the last coin that will complete the payment). This will cause all of the inserted money to be immediately returned. Since the coins the user inserted have already been inserted into the pool, the returned coins will not be the same and in some cases, the machine will change the denominators to those of higher value.

Internally, the machine has two main components, one that is responsible for user interaction (such as product buttons, coin slot, etc.) and another that manages the stock, that is a database with products and available coins for change. 